\section{Type Preservation for References}

\begin{enumerate}[(a)]
\item
\begin{align*}
\texttt{ack} \triangleq ( \lambda ~ &\texttt{refack}:(\ii \rightarrow \ii \rightarrow \ii) ~ \ff .\\
&\texttt{refack} := \lambda m : \ii. ~ \lambda n : \ii . ~ \\
&\texttt{if } m = 0 \texttt{ then } n+1 \texttt{ else}\\
&\texttt{if } m>0 \texttt{ and } n = 0 \texttt{ then } (!\texttt{refack}) ~ (m-1) ~1 \texttt{ else } \\
&(!\texttt{refack}) ~ (m-1) ~ ((!\texttt{refack}) ~ m ~ (n-1)) ) \\
&\left ( \ff \left ( ~ \lambda x : \ii~.~\lambda y : \ii~.~ x+ y \right ) \right )
\end{align*}
The idea is that we first define a variable \texttt{refack} of type $(\ii \rightarrow \ii \rightarrow \ii) ~ \ff$ and update this location with the recursive function of the Ackermann function.
\item The same idea can be used to implement the \texttt{rec} $f : \sigma \rightarrow \tau.\lambda x : \sigma.e$ where we first create a variable $rf$ of type $(\sigma \rightarrow \tau)~\ff$ and replace any occurrence of $f$ in $e$ by $!rf$ and assign $e$ to $rf$.
\item To formalize the type preservation property we will first define a function $f$ to encapsulate how the location context is changed as evaluation rules are applied. 

The idea is for any well-typed expression $e$ given $\Sigma$ and $\Gamma$, if $\Sigma; \Gamma \vdash \sigma$ and $f(
  \langle e, \sigma \rangle \rightarrow \langle e', \sigma' \rangle,
  \Gamma, 
  \Sigma) = \Sigma'$ then $\Sigma'; \Gamma \vdash \sigma'$. 

The function $f$ can be defined inductively based on the proof tree of $\langle e, \sigma \rangle \rightarrow \langle e', \sigma' \rangle$. We define $f$ for each evaluation rule as follows
\begin{enumerate}[i.]
\item If \Rule{}{
    \langle e, \sigma \rangle \rightarrow \langle e', \sigma' \rangle
  }{
    \langle E[e], \sigma \rangle \rightarrow \langle E[e'], \sigma' \rangle
  }

  then $f(
  \langle E[e], \sigma \rangle \rightarrow \langle E[e'], \sigma' \rangle,
  \Gamma, 
  \Sigma) \triangleq f(
  \langle e, \sigma \rangle \rightarrow \langle e', \sigma' \rangle,
  \Gamma, 
  \Sigma)$

\item If \Rule{}{
    l \notin \texttt{dom} ~ \sigma
  }{
    \langle \ff~v, \sigma \rangle \rightarrow \langle l, \sigma[v/l] \rangle
  } and $\Sigma; \Gamma \vdash v : \tau$

  then $f(
  \langle \ff~v, \sigma \rangle \rightarrow \langle l, \sigma[v/l],
  \Gamma, 
  \Sigma) \triangleq 
  \Sigma; l, \tau$

\item If \Rule{}{
  }{
    \langle !l, \sigma \rangle \rightarrow \langle \sigma(l), \sigma \rangle
  }

  then $f(
  \langle !l, \sigma \rangle \rightarrow \langle \sigma(l), \sigma \rangle,
  \Gamma, 
  \Sigma) \triangleq \Sigma$

\item If \Rule{}{
  }{
    \langle l := v, \sigma \rangle \rightarrow \langle v, \sigma[v/l] \rangle
  }

  then $f(
  \langle l := v, \sigma \rangle \rightarrow \langle v, \sigma[v/l],
  \Gamma, 
  \Sigma) \triangleq \Sigma$

\item Other evaluation rules are extended from $\lambda^{\rightarrow}$ so the store $\sigma$ is unchanged.

  $f(
  \langle e, \sigma \rangle \rightarrow \langle e', \sigma,
  \Gamma, 
  \Sigma) \triangleq \Sigma$
\end{enumerate}

Notice that the location context is only changed when a new location is inserted into the context domain.

The type preservation property for the extended $\lambda^{\rightarrow}$:
\begin{align*}
&\Sigma;\Gamma \vdash e : \tau \\
\wedge~&\Sigma;\Gamma \vdash \sigma\\
\wedge~&\langle e, \sigma \rangle \rightarrow \langle e', \sigma' \rangle\\
\wedge~&\Sigma' = f(\langle e, \sigma \rangle \rightarrow \langle e', \sigma' \rangle, \Gamma, \Sigma)\\
\Longrightarrow~&\Sigma';\Gamma \vdash \sigma' \wedge \Sigma';\Gamma \vdash e' : \tau
\end{align*}
\item We make use of the following lemmas:

\textbf{Lemma 1}: If $\Sigma;\Gamma \vdash e : \tau$ and $\Sigma \subseteq \Sigma'$ then $\Sigma';\Gamma \vdash e : \tau$ where $\Sigma \subseteq \Sigma'$ if $\dd \Sigma \subseteq \dd \Sigma'$ and $\forall l \in \dd \Sigma. ~ \Sigma(l) = \Sigma'(l)$.

%%  \textbf{Proof} (of lemma 1): %%TODO

\textbf{Lemma 2}: If $\Sigma;\Gamma \vdash e : \tau \wedge\Sigma;\Gamma \vdash \sigma \wedge \langle e, \sigma \rangle \rightarrow \langle e', \sigma' \rangle$ then $\Sigma \subseteq \Sigma' =  f(\langle e, \sigma \rangle \rightarrow \langle e', \sigma' \rangle, \Gamma, \Sigma)$

We prove the type preservation property above by induction on the proof tree of $\langle e, \sigma \rangle \rightarrow \langle e', \sigma' \rangle$.

\begin{enumerate}[i.]
\item 
  \Rule{}{
    l \notin \texttt{dom} ~ \sigma
  }{
    \langle \ff~v, \sigma \rangle \rightarrow \langle l, \sigma[v/l] \rangle
  }

  First it is easy to see that the type of $\ff~v$ under any location context $\Sigma$ and assumptions $\Gamma$ must be $\tau~\ff$ for some type $\tau$.

  Let $\sigma'$ denote $\sigma[v/l]$. Suppose there exist $\Sigma, \Gamma$ and $\tau$ such that 
  \begin{align*}
  &\Sigma;\Gamma \vdash \ff~v : \tau~\ff \\
  \wedge~&\Sigma;\Gamma \vdash \sigma\\
  \wedge~&\Sigma' = f(\langle \ff~v, \sigma \rangle \rightarrow \langle l, \sigma' \rangle, \Gamma, \Sigma)
  \end{align*}
  We will prove that $\Sigma';\Gamma \vdash \sigma' \wedge \Sigma';\Gamma \vdash l : \tau ~\ff$.

  Since $\Sigma;\Gamma \vdash \ff~v : \tau~\ff$ the last 2 levels of the proof tree of this derivation must be

  \Rulec{}{
    \Sigma;\Gamma \vdash v : \tau
  }{
    \Sigma;\Gamma \vdash \ff~v : \tau ~ \ff
  }

  From the definition of $f$, $\Sigma' = \Sigma; l, \tau \Rightarrow \Sigma \subseteq \Sigma'$

  \begin{itemize}
    \item Claim: $\Sigma';\Gamma \vdash \sigma'$

    We need to prove that $\dd \Sigma' = \dd \sigma'$ and $\forall l' \in \dd \sigma'.~\Sigma';\Gamma \vdash \sigma'(l') : \Sigma'(l')$.
    \begin{itemize}
    \item $l \notin \dd \sigma \Rightarrow \dd \Sigma' = \dd \Sigma \cup \{l\} = \dd \sigma \cup \{l\} = \dd \sigma'$
    \item For any location $l' \in \dd \sigma'$ and $l' \neq l$, $l' \in \dd \sigma \Rightarrow \Sigma; \Gamma \vdash \sigma(l') : \Sigma(l')$. 

    Since $\sigma(l') = \sigma'(l')$ and $\Sigma(l') = \Sigma'(l')$ we have $\Sigma; \Gamma \vdash \sigma'(l') : \Sigma'(l')$. Applying lemma 1 on the expression $\sigma'(l')$, $\Sigma'; \Gamma \vdash \sigma'(l') : \Sigma'(l')$.

    On the other hand for the location $l$, $\sigma'(l) = v$ and $\Sigma;\Gamma \vdash v : \tau$.

    $\tau = \Sigma'(l) \Rightarrow \Sigma;\Gamma \vdash \sigma'(l) : \Sigma'(l)$. Again by lemma 1 $\Sigma';\Gamma \vdash \sigma'(l) : \Sigma'(l)$.
    \end{itemize}
    \item Claim: $\Sigma';\Gamma \vdash l : \tau ~\ff$. This follows directly from $\Sigma'(l) = \tau$.
  \end{itemize}
\item \Rule{}{
  }{
    \langle !l, \sigma \rangle \rightarrow \langle \sigma(l), \sigma \rangle
  }

  Suppose there exist $\Sigma, \Gamma$ and $\tau$ such that 
  \begin{align*}
  &\Sigma;\Gamma \vdash !l : \tau \\
  \wedge~&\Sigma;\Gamma \vdash \sigma\\
  \wedge~&\Sigma' = f(\langle !l, \sigma \rangle \rightarrow \langle \sigma(l), \sigma \rangle, \Gamma, \Sigma)
  \end{align*}
  We will prove that $\Sigma';\Gamma \vdash \sigma \wedge \Sigma';\Gamma \vdash \sigma(l) : \tau$ which is equivalent to $\Sigma;\Gamma \vdash \sigma \wedge \Sigma;\Gamma \vdash \sigma(l) : \tau$ since by definition of $f$, $\Sigma' = \Sigma$.

  The first statement is from the assumptions.

  The proof tree of $\Sigma;\Gamma \vdash !l : \tau$ must be

  \Rulec{}{
  \Rulec{}{
    \Sigma(l) = \tau
  }{
    \Sigma;\Gamma \vdash l : \tau~\ff
  }
  }{
    \Sigma;\Gamma \vdash !l : \tau
  }

  $\Sigma; \Gamma \vdash \sigma \Rightarrow \Sigma;\Gamma \vdash \sigma(l) : \Sigma(l) \Rightarrow \Sigma;\Gamma \vdash \sigma(l) : \tau$

\item \Rule{}{
  }{
    \langle l := v, \sigma \rangle \rightarrow \langle v, \sigma[v/l] \rangle
  }

  Suppose there exist $\Sigma, \Gamma$ and $\tau$ such that 
  \begin{align*}
  &\Sigma;\Gamma \vdash l := v : \tau \\
  \wedge~&\Sigma;\Gamma \vdash \sigma\\
  \wedge~&\Sigma' = f(\langle l := v, \sigma \rangle \rightarrow \langle v, \sigma[v/l] \rangle, \Gamma, \Sigma)
  \end{align*}
  Let $\sigma' = \sigma[v/l]$. We will prove that $\Sigma';\Gamma \vdash \sigma' \wedge \Sigma';\Gamma \vdash v : \tau$ which is equivalent to $\Sigma;\Gamma \vdash \sigma' \wedge \Sigma;\Gamma \vdash v : \tau$ since by definition of $f$, $\Sigma' = \Sigma$.

  \begin{itemize}
  \item Claim: $\Sigma;\Gamma \vdash \sigma'$
    \begin{itemize}
    \item $l \in \dd \sigma\Rightarrow \dd \sigma' = \dd \sigma = \dd \Sigma$
    \item For any arbitrary location $l' \in \dd \sigma'$ such that $l' \neq l$, $\sigma(l') = \sigma'(l')$.

    $\Sigma; \Gamma \vdash \sigma \Rightarrow\Sigma;\Gamma \vdash \sigma(l') : \Sigma(l') \Rightarrow \Sigma; \Gamma \vdash \sigma'(l') : \Sigma(l').$

    On the other hand for the location $l$, $\sigma'(l) = v$. The bottom of the proof tree of $\Sigma; \Gamma \vdash l := v : \tau$ must be:

    \Rulec{}{
    \Rulec{}{\Sigma(l) = \tau
  }{
    \Sigma;\Gamma \vdash l : \tau~\ff
  } \qquad 
    \Sigma;\Gamma \vdash v : \tau
    }{
    \Sigma; \Gamma \vdash l := v : \tau
  }

  Thus $\Sigma; \Gamma \vdash \sigma'(l) : \tau$ since $\Sigma; \Gamma \vdash v : \tau$. 

  $\Sigma(l) = \tau$ so $\Sigma; \Gamma \vdash \sigma'(l) : \Sigma(l)$
  \end{itemize}
  \item Claim: $\Sigma;\Gamma \vdash v : \tau$. This is from the proof tree of $\Sigma; \Gamma \vdash l := v : \tau$ above.
  \end{itemize}
\item 
  \Rule{}{
    \langle e, \sigma \rangle \rightarrow \langle e', \sigma' \rangle
  }{
    \langle E[e], \sigma \rangle \rightarrow \langle E[e'], \sigma' \rangle
  }

  \textbf{Induction hypothesis}: If $\Sigma_1, \Gamma_1$ and $\tau_1$ satisfy
  \begin{align*}
  &\Sigma_1;\Gamma_1 \vdash e : \tau_1 \\
  \wedge~&\Sigma_1;\Gamma_1 \vdash \sigma\\
  \wedge~&\Sigma_1' = f(\langle e, \sigma \rangle \rightarrow \langle e', \sigma' \rangle, \Gamma_1, \Sigma_1)
  \end{align*}
then $\Sigma_1';\Gamma_1 \vdash \sigma' \wedge \Sigma_1';\Gamma_1 \vdash e' : \tau_1$

Now suppose there exist $\Sigma, \Gamma$ and $\tau$ such that
\begin{align*}
&\Sigma;\Gamma \vdash E[e] : \tau \\
\wedge~&\Sigma;\Gamma \vdash \sigma\\
\wedge~&\Sigma' = f(\langle E[e], \sigma \rangle \rightarrow \langle E[e'], \sigma' \rangle, \Gamma, \Sigma)
\end{align*}
We will prove that $\Sigma';\Gamma \vdash \sigma' \wedge \Sigma';\Gamma \vdash E[e'] : \tau$.

By definition of $f$, $\Sigma' = f(\langle e, \sigma \rangle \rightarrow \langle e', \sigma' \rangle, \Gamma, \Sigma)$

\begin{itemize}
\item Claim: $\Sigma';\Gamma \vdash \sigma'$.

Since $E[e]$ is well-typed in $\Sigma; \Gamma$, so is $e$. 

Suppose $\Sigma;\Gamma \vdash e : \tau'$. $\Sigma;\Gamma \vdash \sigma$ so applying the induction hypothesis on $\Sigma, \Gamma$ and $\tau'$ we obtain the claim directly.
\item Claim: $\Sigma';\Gamma \vdash E[e'] : \tau$.

Here are the premises that we make use of in this proof.
\begin{itemize}
  \item $\Sigma' = f(\langle e, \sigma \rangle \rightarrow \langle e', \sigma' \rangle, \Gamma, \Sigma)$
  \item $\Sigma;\Gamma \vdash e : \tau'$ for some type $\tau'$
  \item $\Sigma';\Gamma \vdash e' : \tau'$ 
  \item $\Sigma;\Gamma \vdash E[e] : \tau$ 
\end{itemize}
In order to prove the claim we induct on the context $E$.
\begin{itemize}
  \item Case $E = [.]$ then $E[e] = e$ and thus $\tau = \tau'$ since $e$ is well typed. The claim is equivalent to $\Sigma';\Gamma \vdash e' : \tau'$.
  \item Case $E = E_1~e_1$ then $E[e] = E_1[e]~e_1$. 

  The bottom of the proof tree of $\Sigma;\Gamma \vdash E[e] : \tau$ must be \Rule{}{
    \Sigma;\Gamma \vdash E_1[e] : \tau_1 \rightarrow\tau
    \qquad
    \Sigma;\Gamma \vdash e_1 : \tau_1
  }{
    \Sigma;\Gamma \vdash E_1[e]~ e_1 : \tau
  }

  Using the induction hypothesis on context $E_1$ and $\Sigma;\Gamma \vdash E_1[e] : \tau_1 \rightarrow\tau$ we have \\
  $\Sigma';\Gamma \vdash E_1[e'] : \tau_1 \rightarrow\tau$.

  From lemma 2, $\Sigma \subseteq \Sigma'$ and using lemma 1 $\Sigma' ; \Gamma \vdash e_1 : \tau_1$.

  $\Rightarrow$\Rule{}{
    \Sigma';\Gamma \vdash E_1[e'] : \tau_1 \rightarrow\tau
    \qquad
    \Sigma';\Gamma \vdash e_1 : \tau_1
  }{
    \Sigma';\Gamma \vdash E_1[e']~ e_1 : \tau
  }
  \item Case $E = v~E_1$ then $E[e] = v~E_1[e]$. 

  The bottom of the proof tree of $\Sigma;\Gamma \vdash E[e] : \tau$ must be \Rule{}{
    \Sigma;\Gamma \vdash v : \tau_1 \rightarrow\tau
    \qquad
    \Sigma;\Gamma \vdash E_1[e] : \tau_1
  }{
    \Sigma;\Gamma \vdash v~E_1[e] : \tau
  }

  Using the induction hypothesis on context $E_1$ and $\Sigma;\Gamma \vdash E_1[e] : \tau_1$ \\
  we have $\Sigma';\Gamma \vdash E_1[e'] : \tau_1$.

  From lemma 2, $\Sigma \subseteq \Sigma'$ and using lemma 1 $\Sigma' ; \Gamma \vdash v : \tau_1 \rightarrow \tau$.

  $\Rightarrow$\Rule{}{
    \Sigma';\Gamma \vdash v : \tau_1 \rightarrow\tau
    \qquad
    \Sigma';\Gamma \vdash E_1[e] : \tau_1
  }{
    \Sigma';\Gamma \vdash v~E_1[e'] : \tau
  }
  \item Case $E = \ff~E_1$ then $E[e] = \ff~E_1[e]$ and $\tau = \tau'~\ff$. 

  The bottom of the proof tree of $\Sigma;\Gamma \vdash E[e] : \tau'~\ff$ must be \Rule{}{
    \Sigma;\Gamma \vdash E_1[e] : \tau'
  }{
    \Sigma;\Gamma \vdash \ff~E_1[e] : \tau'~\ff
    }

  Using the induction hypothesis on context $E_1$ and $\Sigma;\Gamma \vdash E_1[e] : \tau'$ \\
  we have $\Sigma';\Gamma \vdash E_1[e'] : \tau'$.

  $\Rightarrow$\Rule{}{
    \Sigma';\Gamma \vdash E_1[e] : \tau'
  }{
    \Sigma';\Gamma \vdash \ff~E_1[e] : \tau'~\ff
    }
  \item Case $E =~!E_1$then $E[e] = ~!E_1[e]$.

  The bottom of the proof tree of $\Sigma;\Gamma \vdash E[e] : \tau$ must be \Rule{}{
    \Sigma;\Gamma \vdash E_1[e] : \tau~\ff
  }{
    \Sigma;\Gamma \vdash !E_1[e] : \tau
    }

  Using the induction hypothesis on context $E_1$ and $\Sigma;\Gamma \vdash E_1[e] : \tau~\ff$ we have $\Sigma';\Gamma \vdash E_1[e] : \tau~\ff$.

  $\Rightarrow$\Rule{}{
    \Sigma';\Gamma \vdash E_1[e] : \tau~\ff
  }{
    \Sigma';\Gamma \vdash !E_1[e] : \tau
    }
  \item Case $E = E_1 := e_1$ then $E[e] = E_1[e] := e_1$. 

  The bottom of the proof tree of $\Sigma;\Gamma \vdash E[e] : \tau$ must be \Rule{}{
    \Sigma;\Gamma \vdash E_1[e] : \tau~\ff
    \qquad
    \Sigma;\Gamma \vdash e_1 : \tau
  }{
    \Sigma;\Gamma \vdash E_1[e] := e_1 : \tau
  }

  Using the induction hypothesis on context $E_1$ and $\Sigma;\Gamma \vdash E_1[e] : \tau~\ff$ we have $\Sigma';\Gamma \vdash E_1[e] : \tau~\ff$.

  From lemma 2, $\Sigma \subseteq \Sigma'$ and using lemma 1 $\Sigma' ;\Gamma \vdash e_1 : \tau$.

  $\Rightarrow$\Rule{}{
    \Sigma';\Gamma \vdash E_1[e] : \tau~\ff
    \qquad
    \Sigma';\Gamma \vdash e_1 : \tau
  }{
    \Sigma';\Gamma \vdash E_1[e] := e_1 : \tau
  }
  \item Case $E = v := E_1$ then $E[e] = v := E_1[e]$. 

  The bottom of the proof tree of $\Sigma;\Gamma \vdash E[e] : \tau$ must be \Rule{}{
    \Sigma;\Gamma \vdash v : \tau ~ \ff
    \qquad
    \Sigma;\Gamma \vdash E_1[e] : \tau
  }{
    \Sigma;\Gamma \vdash v := E_1[e] : \tau
  }

  Using the induction hypothesison context $E_1$ and $\Sigma;\Gamma \vdash E_1[e] : \tau$ \\
  we have $\Sigma';\Gamma \vdash E_1[e'] : \tau$.

  From lemma 2, $\Sigma \subseteq \Sigma'$ and using lemma 1 $\Sigma' ; \Gamma \vdash v : \tau ~ \ff$.

  $\Rightarrow$\Rule{}{
    \Sigma';\Gamma \vdash v : \tau ~ \ff
    \qquad
    \Sigma';\Gamma \vdash E_1[e] : \tau
  }{
    \Sigma';\Gamma \vdash v := E_1[e] : \tau
  }
\end{itemize} 
\end{itemize}

%%TODO: prove the type of E[e']
\end{enumerate}
\end{enumerate}
