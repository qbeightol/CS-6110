\section{Type Preservation for References}

\begin{enumerate}[(a)]
\item
\begin{align*}
\texttt{ack} \triangleq ( \lambda ~ &\texttt{refack}:(\ii \rightarrow \ii \rightarrow \ii) ~ \ff .\\
&\texttt{refack} := \lambda m : \ii. ~ \lambda n : \ii . ~ \\
&\texttt{if } m = 0 \texttt{ then } n+1 \texttt{ else}\\
&\texttt{if } m>0 \texttt{ and } n = 0 \texttt{ then } (!\texttt{refack}) ~ (m-1) ~1 \texttt{ else } \\
&(!\texttt{refack}) ~ (m-1) ~ ((!\texttt{refack}) ~ m ~ (n-1)) ) \\
&\left ( \ff \left ( ~ \lambda x : \ii~.~\lambda y : \ii~.~ x+ y \right ) \right )
\end{align*}
The idea is that we first define a variable \texttt{refack} of type $(\ii \rightarrow \ii \rightarrow \ii) ~ \ff$ and update this location with the recursive function of the Ackermann function.
\item The same idea can be used to implement the \texttt{rec} $f : \sigma \rightarrow \tau.\lambda x : \sigma.e$ where we first create a variable $rf$ of type $(\sigma \rightarrow \tau)~\ff$ and replace any occurrence of $f$ in $e$ by $!rf$ and assign $e$ to $rf$.
\item To formalize the type preservation property we will first define a function $f$ to encapsulate how the location context is changed as evaluation rules are applied. 

The idea is for any well-types expression $e$ if $\Sigma; \Gamma \vdash \sigma$ and $f(
  \langle e, \sigma \rangle \rightarrow \langle e', \sigma' \rangle,
  \Gamma, 
  \Sigma) = \Sigma'$ where then $\Sigma'; \Gamma \vdash \sigma'$. 

The function $f$ can be defined inductively based on the proof tree of $\langle e, \sigma \rangle \rightarrow \langle e', \sigma' \rangle$. We define $f$ for each evaluation rule as follows
\begin{enumerate}[i.]
\item If \Rule{}{
    \langle e, \sigma \rangle \rightarrow \langle e', \sigma' \rangle
  }{
    \langle E[e], \sigma \rangle \rightarrow \langle E[e'], \sigma' \rangle
  }

  then $f(
  \langle E[e], \sigma \rangle \rightarrow \langle E[e'], \sigma' \rangle,
  \Gamma, 
  \Sigma) \triangleq f(
  \langle e, \sigma \rangle \rightarrow \langle e', \sigma' \rangle,
  \Gamma, 
  \Sigma)$

\item If \Rule{}{
    l \notin \texttt{dom} ~ \sigma
  }{
    \langle \ff~v, \sigma \rangle \rightarrow \langle l, \sigma[v/l] \rangle
  } and $\Sigma; \Gamma \vdash v : \tau$

  then $f(
  \langle \ff~v, \sigma \rangle \rightarrow \langle l, \sigma[v/l],
  \Gamma, 
  \Sigma) \triangleq 
  \Sigma; l, \tau$

\item If \Rule{}{
  }{
    \langle !l, \sigma \rangle \rightarrow \langle \sigma(l), \sigma \rangle
  }

  then $f(
  \langle !l, \sigma \rangle \rightarrow \langle \sigma(l), \sigma \rangle,
  \Gamma, 
  \Sigma) \triangleq \Sigma$

\item If \Rule{}{
  }{
    \langle l := v, \sigma \rangle \rightarrow \langle v, \sigma[v/l] \rangle
  }

  then $f(
  \langle l := v, \sigma \rangle \rightarrow \langle v, \sigma[v/l],
  \Gamma, 
  \Sigma) \triangleq \Sigma$

\item Other evaluation rules are extended from $\lambda^{\rightarrow}$ so the store $\sigma$ is unchanged.

  $f(
  \langle e, \sigma \rangle \rightarrow \langle e', \sigma,
  \Gamma, 
  \Sigma) \triangleq \Sigma$

The type preservation property for the extended $\lambda^{\rightarrow}$:
\begin{align*}
&\Sigma;\Gamma \vdash e : \tau \\
\wedge~&\Sigma;\Gamma \vdash \sigma\\
\wedge~&\langle e, \sigma \rangle \rightarrow \langle e', \sigma' \rangle\\
\wedge~&\Sigma' = f(\langle e, \sigma \rangle \rightarrow \langle e', \sigma' \rangle, \Gamma, \Sigma)\\
\Longrightarrow~&\Sigma';\Gamma \vdash e' : \tau \wedge \Sigma';\Gamma \vdash \sigma'
\end{align*}
\end{enumerate}
\item
\end{enumerate}
