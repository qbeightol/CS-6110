\section{Algorithmic Subtyping}
\newcommand{\subtype}[2]{\vdash #1 \leq #2}
\newcommand{\subtypeNoRefl}[2]{\vdash_\text{\sout{\sf refl}} #1 \leq #2}
\newcommand{\subtypeNoTrans}[2]{\vdash_\text{\sout{\sf trans}} #1 \leq #2}
\newcommand{\asub}{\ensuremath{\vdash\!\!\!\!\raisebox{-.6ex}{\tikz\node[rotate=90]{\footnotesize$\blacktriangledown$};}}}
\newcommand{\algoSubtype}[2]{\asub #1 \leq #2}

Let $\subtypeNoRefl \sigma \tau$ mean that you can derive $\sigma \leq \tau$
without using the reflexivity rule, and let $\subtypeNoTrans \sigma \tau$ mean the same
thing but with the transitivity rule excluded instead of the reflexivity rule.

\begin{enumerate}[(a)]
\item
    \begin{lemma} $\forall \tau . \subtypeNoRefl \tau \tau$ \end{lemma}
    \begin{proof} By induction on the form of $\tau$.

    \underline{$\tau = \top$:} this case follows from the $\subtypeNoRefl \tau \top$
    rule.

    \underline{$\tau = \tau_1 \rightarrow \tau_2$:} By the induction hypothesis,
    $\subtypeNoRefl{\tau_1}{\tau_1}$ and $\subtypeNoRefl{\tau_2}{\tau_2}$
    Applying the subtyping rule for _functions_ then gives
    $\subtypeNoRefl \tau \tau $.

    \underline{$\tau = \tau_1 \times \tau_2$:} By the induction hypothesis,
    $\subtypeNoRefl{\tau_1}{\tau_1}$ and $\subtypeNoRefl{\tau_2}{\tau_2}$
    Applying the subtyping rule for _products_ then gives
    $\subtypeNoRefl \tau \tau$.

    \end{proof}

    Note that we can essentially re-use this proof to show \\

    \begin{lemma} If $\subtypeNoTrans{\tau}{\tau}$ then $\algoSubtype{\tau}{\tau}$ \end{lemma}

    \begin{proof} By induction on the structure of $\tau$. Since all three subtyping rules mentioned in the proof
    above also exist in the type system without reflexivity and transivity, the proof also holds here.
    \end{proof}

\item
    \begin{lemma} If $\subtype{\tau_1}{\tau_2}$ then $\subtypeNoTrans{\tau_1}{\tau_2}$. \end{lemma}

    \begin{proof} We'll use well founded induction on the set of all derivations of $\subtype{\tau_1}{\tau_2}$ where
    $D \prec E$ when $D$ is a subtyping derivation whose height is less than the height of $E$. Our induction hypothesis
    allows us to immediately prove the lemma for all cases of the final rule for deriving $\subtype{\tau_1}{\tau_2}$
    except for the transitivity case:

    $$\inferrule{\subtype{\tau_1}{\tau_3} \\ \subtype{\tau_3}{\tau_2}}{\subtype{\tau_1}{\tau_2}}$$

    % case where $\subtype{\tau_1}{\tau_3}$ and $\subtype{\tau_3}{\tau_2}$.
    Hence, this proof will focus on that case and do a sub-case analysis on the form of $\tau_1$.

    \underline{$\tau_1 = \top$}. In this case, $\tau_2$ and $\tau_3$ must also be top. There's no way to show that
    something other than $\top$ is a subtype of $\top$. Instead of using transivity to show that $\tau_1$ is a subtype
    of $\tau_3$ you could use $\tau \leq \top$ rule directly.

    \underline{$\tau_1 = \tau^\textsf{in}_1 \rightarrow \tau^\textsf{out}_1$:} Here, $\tau_2$ must either be $\top$ or
    a function type. In the former case it isn't necessary to use transitivity---you can rewrite the proof to use
    the $\tau \leq \top$ rule instead. If $\tau_2 = \tau^\textsf{in}_2 \rightarrow \tau^\textsf{out}_2$ then
    $\tau_3 = \tau^\textsf{in}_3 \rightarrow \tau^\textsf{out}_3$. That implies that the derivation is really

    \newcommand{\leftSubtree}{
        \inferrule{
            \subtype{\tau^\textsf{in}_3}{\tau^\textsf{in}_1} \\
            \subtype{\tau^\textsf{out}_1}{\tau^\textsf{out}_3}
        }{
            \subtype{\tau^\textsf{in}_1 \rightarrow \tau^\textsf{out}_1}{\tau^\textsf{in}_3 \rightarrow \tau^\textsf{out}_3}
        }
    }

    \newcommand{\rightSubtree}{
        \inferrule{
            \subtype{\tau^\textsf{in}_2}{\tau^\textsf{in}_3} \\
            \subtype{\tau^\textsf{out}_3}{\tau^\textsf{out}_2}
        }{
            \subtype{\tau^\textsf{in}_3 \rightarrow \tau^\textsf{out}_3}{\tau^\textsf{in}_2 \rightarrow \tau^\textsf{out}_2}
        }
    }

    \newcommand{\base}{
        \subtype{\tau^\textsf{in}_1 \rightarrow \tau^\textsf{out}_1}{\tau^\textsf{in}_2 \rightarrow \tau^\textsf{out}_2}
    }

    $$\inferrule{\leftSubtree \\ \rightSubtree}{\base}$$

    Note that we could use our assumptions to build the following derivation:

    $$
    \inferrule{
        \subtype{\tau^\textsf{in}_3}{\tau^\textsf{in}_1} \\
        \subtype{\tau^\textsf{in}_2}{\tau^\textsf{in}_3}
    }{
        \subtype{\tau^\textsf{in}_2}{\tau^\textsf{in}_1} \\
    }
    $$

    along with a similar derivation for $\subtype{\tau^\textsf{out}_1}{\tau^\textsf{out}_2}$. Since these derivations
    have a smaller height than the derivation of $\subtype{\tau_1}{\tau_2}$, we can apply our inductive hypothesis to
    conclude

    $$\subtypeNoTrans{\tau^\textsf{in}_2}{\tau^\textsf{in}_1}$$
    $$\subtypeNoTrans{\tau^\textsf{out}_1}{\tau^\textsf{out}_2}$$

    Therefore, $\subtypeNoTrans{\tau_1}{\tau_2}$ is derivable without using transivity:

    $$
    \inferrule{
        \subtypeNoTrans{\tau^\textsf{in}_2}{\tau^\textsf{in}_1}
        \subtypeNoTrans{\tau^\textsf{out}_1}{\tau^\textsf{out}_2}
    }{
        \subtypeNoTrans{\tau^\textsf{in}_1 \rightarrow \tau^\textsf{out}_1}{\tau^\textsf{in}_2 \rightarrow \tau^\textsf{out}_2}
    }
    $$

    % Note that by the induction hypothesis, $\subtypeNoTrans{\tau_1}{\tau_2}$

    \underline{$\tau_1 = \tau^\ell_1 \times \tau^r_1$:} Here, $\tau_2$ must either be top or a product type, but
    as mentioned before, the $\tau_2 = \top$ case can be handled using the $\tau \leq \top$ rule. In the case where
    $\tau_2 = \tau^\ell_2 \times \tau^r_2$, $\tau_3$ must also be a product. That implies that the derivation has
    this form

    \renewcommand{\leftSubtree}{
        \inferrule{
            \subtype{\tau^\ell_1}{\tau^\ell_3} \\
            \subtype{\tau^r_1}{\tau^r_3}
        }{
            \subtype{\tau^\ell_1 \times \tau^r_1}{\tau^\ell_3 \times \tau^r_3}
        }
    }

    \renewcommand{\rightSubtree}{
        \inferrule{
            \subtype{\tau^\ell_3}{\tau^\ell_2} \\
            \subtype{\tau^r_3}{\tau^r_2}
        }{
            \subtype{\tau^\ell_3 \times \tau^r_3}{\tau^\ell_2 \times \tau^r_2}
        }
    }

    \renewcommand{\base}{
        \subtype{\tau^\ell_1 \times \tau^r_1}{\tau^\ell_2 \times \tau^r_2}
    }

    $$\inferrule{\leftSubtree \\ \rightSubtree}{\base}$$

    We can use the sub-derivations in our tree to build derivations for $\subtype{\tau^\ell_1}{\tau^\ell_2}$
    and $\subtype{\tau^r_1}{\tau^r_2}$. Then we can apply the IH to these derivations to conclude
    $\subtypeNoTrans{\tau^\ell_1}{\tau^\ell_2}$ and $\subtypeNoTrans{\tau^r_1}{\tau^r_2}$. Finally, we can derive
    the desired result:

    $$
    \inferrule{
        \subtypeNoTrans{\tau^\ell_1}{\tau^\ell_2}
        \subtypeNoTrans{\tau^r_1}{\tau^r_2}
    }{
        \subtypeNoTrans{\tau^\ell_1 \times \tau^r_2}{\tau^\ell_2 \times \tau^r_2}
    }
    $$

    \end{proof}
\item
    \begin{theorem} If $\subtype{\tau_1}{\tau_2}$ then $\algoSubtype{\tau_1}{\tau_2}$. \end{theorem}
    \begin{proof} By our lemmas, $\subtype{\tau_1}{\tau_2}$ implies $\subtypeNoTrans{\tau_1}{\tau_2}$ and
    $\subtypeNoTrans{\tau_1}{\tau_2}$ imples $\algoSubtype{\tau_1}{\tau_2}$. Hence $\subtype{\tau_1}{\tau_2}$
    implies $\algoSubtype{\tau_1}{\tau_2}$
    \end{proof}
\end{enumerate}
