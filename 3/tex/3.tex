\documentclass[10pt]{article}
\usepackage{fancyhdr}
\usepackage{enumerate}
\usepackage{geometry}
\usepackage{parskip}
\usepackage{sectsty}

% fancyhdr %%%%%%%%%%%%%%%%%%%%%%%%%%%%%%%%%%%%%%%%%%%%%%%%%%%%%%%%%%%%%%%%%%%%%
\pagestyle{fancy}

\lhead{\sf CS 6110 Homework 2}
\rhead{\sf Giang Nguyen (htn26) and Quinn Beightol (qeb2)}

% sectsty %%%%%%%%%%%%%%%%%%%%%%%%%%%%%%%%%%%%%%%%%%%%%%%%%%%%%%%%%%%%%%%%%%%%%%
\allsectionsfont{\sf}

% Macros %%%%%%%%%%%%%%%%%%%%%%%%%%%%%%%%%%%%%%%%%%%%%%%%%%%%%%%%%%%%%%%%%%%%%%%

\newcommand{\stepsTo}{\rightarrow}
\newcommand{\multiStepsTo}{\stepsTo^*}

% Document %%%%%%%%%%%%%%%%%%%%%%%%%%%%%%%%%%%%%%%%%%%%%%%%%%%%%%%%%%%%%%%%%%%%%

\begin{document}
\section{Static and Dynamic Scope}%%%%%%%%%%%%%%%%%%%%%%%%%%%%%%%%%%%%%%%%%%%%%%
\begin{enumerate}[(a)]
    \item static scope: 21
    \item dynamic scope: 42
\end{enumerate}
The difference arises because of how $x$ is interpretted in each scheme. Using
static scope, $f$ always interprets $x$ as 3 and $g$ always interprets $x$ as 4.
Hence, evaluating $g~y$ yields $5 + 3 + 3 = 11$, evaluating $f~x$ yields
$7 + 3$ ($x$ gets bound to 7 shortly before the $f~x$ expression), and evaluating
their sum yields 21. Using dynamic scope, however, causes $f~x$ to evalute to
$14$ because $f$ using the most recent $x = 7$ binding, and causes $g~y$ to
evaluate to 28 because $g$ uses $x = 7$ during it's first invocation of $f$ and
$x = 14$ during the second invocation. $f~x + g~y$ is then 42.

\section{Mutual Fixpoints}%%%%%%%%%%%%%%%%%%%%%%%%%%%%%%%%%%%%%%%%%%%%%%%%%%%%%%
We solved this problem by first doing the karma and then instantiating our
karma solution in the case where $n = 2$. Consequently, we'll present this
section out of order so that it's easier to undestand our solution to part (a).

We'll abuse $\overline{e}$ to generally denote some expression $e$ repeated $n$
times, where the expansion may involve a subscript $i$.

Specifically, let $\overline{x}$ be a shorthand for $x_1~x_2~\ldots~x_n$. So $f~\overline{x}$
denotes $f~x_1~x_2~\ldots~x_n$ and $\lambda \overline{x}.~e$ denotes $\lambda x_1 x_2~\ldots~x_n.~e$

Let $\overline{{f \overline{f}}} = (f_1 \overline{f})~(f_2 \overline{f})~\ldots~(f_n \overline{f})$.

Let $\overline{\lambda \overline{f}.~t~(\overline{f~\overline{f}})}
= (\lambda \overline{f}.~t_1~(\overline{f~\overline{f}}))
  ~(\lambda \overline{f}.~t_2~(\overline{f~\overline{f}}))
  ~\ldots
  ~(\lambda \overline{f}.~t_n~(\overline{f~\overline{f}}))$

Then we can write the generalized Y-combinator this way:

$$Y^n_i= \lambda \overline{t}.
    ~(\lambda \overline{f}.~t_i~(\overline{f_i~\overline{f}}))
    ~\overline{\lambda \overline{f}.~t~(\overline{f~\overline{f}})} $$

where the $n$ in $Y^n_i$ specificies how many times an expression should be
repeated.

Instantiating $Y^n_i$ for the $n = 2$ case gives these two combinators:

\begin{eqnarray*}
    Y^2_1 & = \lambda t_1 t_2 . & (\lambda f g.~t_1~(f~f~g)~(g~f~g)) \\
          &                     & (\lambda f g.~t_1~(f~f~g)~(g~f~g)) \\
          &                     & (\lambda f g.~t_2~(f~f~g)~(g~f~g))
\end{eqnarray*}

\begin{eqnarray*}
    Y^2_1 & = \lambda t_1 t_2 . & (\lambda f g .~t_2~(f~f~g)~(g~f~g)) \\
          &                     & (\lambda f g .~t_1~(f~f~g)~(g~f~g)) \\
          &                     & (\lambda f g .~t_2~(f~f~g)~(g~f~g))
\end{eqnarray*}

Passing in $F$ and $G$ to each combinator yields the following:

\begin{eqnarray*}
    Y^2_1~F~G & =             & (\lambda f g .~F~(f~f~g)~(g~f~g)) \\
              &               & (\lambda f g .~F~(f~f~g)~(g~f~g)) \\
              &               & (\lambda f g .~G~(f~f~g)~(g~f~g)) \\
              & \multiStepsTo & F~((\lambda f g .~F~(f~f~g)~(g~f~g)) \\
              &               & \hspace{.45cm}(\lambda f g .~F~(f~f~g)~(g~f~g)) \\
              &               & \hspace{.45cm}(\lambda f g .~G~(f~f~g)~(g~f~g))) \\
              &               & \hspace{.35cm}((\lambda f g .~G~(f~f~g)~(g~f~g)) \\
              &               & \hspace{.45cm}(\lambda f g .~F~(f~f~g)~(g~f~g)) \\
              &               & \hspace{.45cm}(\lambda f g .~G~(f~f~g)~(g~f~g)))
\end{eqnarray*}

Meanwhile,

\begin{eqnarray*}
    Y^2_1~F~G & =             & (\lambda f g .~F~(f~f~g)~(g~f~g)) \\
              &               & (\lambda f g .~F~(f~f~g)~(g~f~g)) \\
              &               & (\lambda f g .~G~(f~f~g)~(g~f~g)) \\
              & \multiStepsTo & F~((\lambda f g .~G~(f~f~g)~(g~f~g)) \\
              &               & \hspace{.45cm}(\lambda f g .~F~(f~f~g)~(g~f~g)) \\
              &               & \hspace{.45cm}(\lambda f g .~G~(f~f~g)~(g~f~g))) \\
              &               & \hspace{.35cm}((\lambda f g .~G~(f~f~g)~(g~f~g)) \\
              &               & \hspace{.45cm}(\lambda f g .~F~(f~f~g)~(g~f~g)) \\
              &               & \hspace{.45cm}(\lambda f g .~G~(f~f~g)~(g~f~g)))
\end{eqnarray*}

By comparing the definitions of $Y^2_i~F~G$ with the results of partially
evaluating $Y^2_i~F~G$, we can see that

$$Y^2_1~F~G = F~(Y^2_1~F~G)~(Y^2_2~F~G) $$

and

$$Y^2_1~F~G = G~(Y^2_1~F~G)~(Y^2_2~F~G) $$

\section{State}%%%%%%%%%%%%%%%%%%%%%%%%%%%%%%%%%%%%%%%%%%%%%%%%%%%%%%%%%%%%%%%%%
\begin{enumerate}
    \item
    \item
    \item
\end{enumerate}
\section{Control Flow}%%%%%%%%%%%%%%%%%%%%%%%%%%%%%%%%%%%%%%%%%%%%%%%%%%%%%%%%%%
\begin{enumerate}
    \item
    \item
    \item
\end{enumerate}
\section{Closure Conversion}%%%%%%%%%%%%%%%%%%%%%%%%%%%%%%%%%%%%%%%%%%%%%%%%%%%%
See \texttt{lifting.zip}.
\section{Debriefing}%%%%%%%%%%%%%%%%%%%%%%%%%%%%%%%%%%%%%%%%%%%%%%%%%%%%%%%%%%%%
\begin{enumerate}
    \item
    \item
    \item
    \item
    \item
\end{enumerate}
\end{document}
