\documentclass[10pt]{article}
\usepackage{fancyhdr, amsmath, amssymb, stmaryrd}
\usepackage{enumerate}
\usepackage{geometry}
\usepackage{parskip}
\usepackage{sectsty}
\usepackage{mathpartir}


% fancyhdr %%%%%%%%%%%%%%%%%%%%%%%%%%%%%%%%%%%%%%%%%%%%%%%%%%%%%%%%%%%%%%%%%%%%%
\pagestyle{fancy}

\lhead{\sf CS 6110 Homework 3}
\rhead{\sf Giang Nguyen (htn26) and Quinn Beightol (qeb2)}

% sectsty %%%%%%%%%%%%%%%%%%%%%%%%%%%%%%%%%%%%%%%%%%%%%%%%%%%%%%%%%%%%%%%%%%%%%%
\allsectionsfont{\sf}

% Macros %%%%%%%%%%%%%%%%%%%%%%%%%%%%%%%%%%%%%%%%%%%%%%%%%%%%%%%%%%%%%%%%%%%%%%%

\newcommand{\stepsTo}{\rightarrow}
\newcommand{\multiStepsTo}{\stepsTo^*}
\newcommand{\Rule}[3]{
    \label{rule:#1}
  \hfill
  \ensuremath{\inferrule{#2}{#3}}
  \hfill
}
\newcommand{\RuleWithCondition}[4]{
  \label{rule:#1}
  \hfill
  \ensuremath{\inferrule{#2}{#3}}
  (#4)
  \hfill
}

% Document %%%%%%%%%%%%%%%%%%%%%%%%%%%%%%%%%%%%%%%%%%%%%%%%%%%%%%%%%%%%%%%%%%%%%

\begin{document}
\section{Static and Dynamic Scope}%%%%%%%%%%%%%%%%%%%%%%%%%%%%%%%%%%%%%%%%%%%%%%
\begin{enumerate}[(a)]
    \item static scope: 21
    \item dynamic scope: 42
\end{enumerate}
The difference arises because of how $x$ is interpretted in each scheme. Using
static scope, $f$ always interprets $x$ as 3 and $g$ always interprets $x$ as 4.
Hence, evaluating $g~y$ yields $5 + 3 + 3 = 11$, evaluating $f~x$ yields
$7 + 3$ ($x$ gets bound to 7 shortly before the $f~x$ expression), and evaluating
their sum yields 21. Using dynamic scope, however, causes $f~x$ to evalute to
$14$ because $f$ using the most recent $x = 7$ binding, and causes $g~y$ to
evaluate to 28 because $g$ uses $x = 7$ during it's first invocation of $f$ and
$x = 14$ during the second invocation. $f~x + g~y$ is then 42.

\section{Mutual Fixpoints}%%%%%%%%%%%%%%%%%%%%%%%%%%%%%%%%%%%%%%%%%%%%%%%%%%%%%%
We solved this problem by first doing the karma and then instantiating our
karma solution in the case where $n = 2$. Consequently, we'll present this
section out of order so that it's easier to undestand our solution to part (a).

We'll abuse $\overline{e}$ to generally denote some expression $e$ repeated $n$
times, where the expansion may involve a subscript $i$.

Specifically, let $\overline{x}$ be a shorthand for $x_1~x_2~\ldots~x_n$. So $f~\overline{x}$
denotes $f~x_1~x_2~\ldots~x_n$ and $\lambda \overline{x}.~e$ denotes $\lambda x_1 x_2~\ldots~x_n.~e$

Let $\overline{{f \overline{f}}} = (f_1 \overline{f})~(f_2 \overline{f})~\ldots~(f_n \overline{f})$.

Let $\overline{\lambda \overline{f}.~t~(\overline{f~\overline{f}})}
= (\lambda \overline{f}.~t_1~(\overline{f~\overline{f}}))
  ~(\lambda \overline{f}.~t_2~(\overline{f~\overline{f}}))
  ~\ldots
  ~(\lambda \overline{f}.~t_n~(\overline{f~\overline{f}}))$

Then we can write the generalized Y-combinator this way:

$$Y^n_i= \lambda \overline{t}.
    ~(\lambda \overline{f}.~t_i~(\overline{f_i~\overline{f}}))
    ~\overline{\lambda \overline{f}.~t~(\overline{f~\overline{f}})} $$

where the $n$ in $Y^n_i$ specificies how many times an expression should be
repeated.

Instantiating $Y^n_i$ for the $n = 2$ case gives these two combinators:

\begin{eqnarray*}
    Y^2_1 & = \lambda t_1 t_2 . & (\lambda f g.~t_1~(f~f~g)~(g~f~g)) \\
          &                     & (\lambda f g.~t_1~(f~f~g)~(g~f~g)) \\
          &                     & (\lambda f g.~t_2~(f~f~g)~(g~f~g))
\end{eqnarray*}

\begin{eqnarray*}
    Y^2_1 & = \lambda t_1 t_2 . & (\lambda f g .~t_2~(f~f~g)~(g~f~g)) \\
          &                     & (\lambda f g .~t_1~(f~f~g)~(g~f~g)) \\
          &                     & (\lambda f g .~t_2~(f~f~g)~(g~f~g))
\end{eqnarray*}

Passing in $F$ and $G$ to each combinator yields the following:

\begin{eqnarray*}
    Y^2_1~F~G & =             & (\lambda f g .~F~(f~f~g)~(g~f~g)) \\
              &               & (\lambda f g .~F~(f~f~g)~(g~f~g)) \\
              &               & (\lambda f g .~G~(f~f~g)~(g~f~g)) \\
              & \multiStepsTo & F~((\lambda f g .~F~(f~f~g)~(g~f~g)) \\
              &               & \hspace{.45cm}(\lambda f g .~F~(f~f~g)~(g~f~g)) \\
              &               & \hspace{.45cm}(\lambda f g .~G~(f~f~g)~(g~f~g))) \\
              &               & \hspace{.35cm}((\lambda f g .~G~(f~f~g)~(g~f~g)) \\
              &               & \hspace{.45cm}(\lambda f g .~F~(f~f~g)~(g~f~g)) \\
              &               & \hspace{.45cm}(\lambda f g .~G~(f~f~g)~(g~f~g)))
\end{eqnarray*}

Meanwhile,

\begin{eqnarray*}
    Y^2_1~F~G & =             & (\lambda f g .~F~(f~f~g)~(g~f~g)) \\
              &               & (\lambda f g .~F~(f~f~g)~(g~f~g)) \\
              &               & (\lambda f g .~G~(f~f~g)~(g~f~g)) \\
              & \multiStepsTo & F~((\lambda f g .~G~(f~f~g)~(g~f~g)) \\
              &               & \hspace{.45cm}(\lambda f g .~F~(f~f~g)~(g~f~g)) \\
              &               & \hspace{.45cm}(\lambda f g .~G~(f~f~g)~(g~f~g))) \\
              &               & \hspace{.35cm}((\lambda f g .~G~(f~f~g)~(g~f~g)) \\
              &               & \hspace{.45cm}(\lambda f g .~F~(f~f~g)~(g~f~g)) \\
              &               & \hspace{.45cm}(\lambda f g .~G~(f~f~g)~(g~f~g)))
\end{eqnarray*}

By comparing the definitions of $Y^2_i~F~G$ with the results of partially
evaluating $Y^2_i~F~G$, we can see that

$$Y^2_1~F~G = F~(Y^2_1~F~G)~(Y^2_2~F~G) $$

and

$$Y^2_1~F~G = G~(Y^2_1~F~G)~(Y^2_2~F~G) $$

\section{State}%%%%%%%%%%%%%%%%%%%%%%%%%%%%%%%%%%%%%%%%%%%%%%%%%%%%%%%%%%%%%%%%%
\begin{enumerate} [(a)]
    \item $loc$ can be defined recursively as follows:
    \begin{align*}
      loc (l) &\triangleq \{l\}\\
      loc (n) &\triangleq \varnothing\\  
      loc (x) &\triangleq \varnothing\\
      loc (\texttt{ref e}) &\triangleq loc(e)\\
      loc (\texttt{!e}) &\triangleq loc(e)\\
      loc (e_1 := e_2) &\triangleq loc(e_1) \cup loc(e_2)\\
      loc (\texttt{null}) &\triangleq \varnothing\\
      loc (\lambda x.e) &\triangleq loc(e)\\
      loc (e_0;~e_1) &\triangleq loc(e_0) \cup loc(e_1)\\
      loc (\texttt{let } x~=~e_1 \texttt{in } e_2) &\triangleq loc(e_1) \cup loc(e_2)\\
      loc ((e_1, e_2)) &\triangleq loc(e_1) \cup loc(e_2)\\
      loc (\#n~e) &\triangleq loc(e)
    \end{align*}
    \item 
    \begin{align*}
    &\langle \lambda x.((!l_1)~2)~(\texttt{ref 1}),~\{\}[l_1 \mapsto \lambda y. \texttt{ref y}]\rangle\\
    \rightarrow &\langle \lambda x.((!l_1)~2)~l_2,~\{\}[l_1 \mapsto \lambda y. \texttt{ref y},~l_2 \mapsto 1]\rangle\\
    \rightarrow &\langle (!l_1)~2, \{\}[l_1 \mapsto \lambda y. \texttt{ref y},~l_2 \mapsto 1]\rangle\\
    \rightarrow &\langle (\lambda y.\texttt{ref y})~2,~\{\}[l_1 \mapsto \lambda y. \texttt{ref y},~l_2 \mapsto 1]\rangle\\
    \rightarrow &\langle \texttt{ref 2},~\{\}[l_1 \mapsto \lambda y. \texttt{ref y},~l_2 \mapsto 1]\rangle\\
    \rightarrow &\langle l_3,~\{\}[l_1 \mapsto \lambda y. \texttt{ref y},~l_2 \mapsto 1,~l_3 \mapsto 2]\rangle
    \end{align*}
    \item We will make use of evaluation contexts in this proof. To recap evaluation contexts can be defined for the fragment of the FL! language as follows:

    \texttt{E ::= [.] | ref E | !E | E:= e | v:=E | E;e | v;E | E e | v E \\
    | let x = E in e | let x = v in E | (E, e) | (v, E) | \#n E}

    $loc$ can also be defined on evaluation contexts as follows:

    \begin{align*}
    loc([.]) &\triangleq \varnothing\\
    loc(\texttt{ref E}) &\triangleq loc(E)\\
    loc(!E) &\triangleq loc(E)\\
    loc(E:=e) &\triangleq loc(E) \cup loc(e)\\
    loc(v:=E) &\triangleq loc(v) \cup loc(E)\\
    loc(E;e) &\triangleq loc(E) \cup loc(e)\\
    loc(v;E) &\triangleq loc(v) \cup loc(E)\\
    loc(E~e) &\triangleq loc(E)\cup loc(e)\\
    loc(v~E) &\triangleq loc(v) \cup loc(E)\\
    loc(\texttt{let x = E in e}) &\triangleq loc(E) \cup loc(e)\\
    loc(\texttt{let x = v in E}) &\triangleq loc(v) \cup loc(E)\\
    loc((E,e)) &\triangleq loc(E) \cup loc(e)\\
    loc((v,E)) &\triangleq loc(v) \cup loc(E)\\
    loc(\#n~E) &\triangleq loc(E)
    \end{align*}

    \textbf{Claim 1:} $loc(E[e]) = loc(E) \cup loc(e)$

    \textbf{Proof:} This claim can be proved by induction on evaluation context $E$. 

    \underline{Case 1}: $E = [.]$ then $loc(E[e]) = loc(e) = loc(e) \cup loc([.])$

    In the following cases we assume that the claim holds for $E$ and prove it for $E'$.

    \underline{Case 2}: $E = \texttt{ref E'}$

    $loc(E[e]) = loc(\texttt{ref E'[e]}) = loc(E'[e]) = loc(E') \cup loc(e) = loc(E) \cup loc(e).$

    \underline{Case 3}: $E = !E'$

    $loc(E[e]) = loc(!E'[e]) = loc(E'[e]) = loc(E') \cup loc(e) = loc(E) \cup loc(e).$

    \underline{Case 4}: $E = E':=e'$

    $loc(E[e]) = loc(E'[e]:=e') = loc(E'[e]) \cup loc(e') = loc(E') \cup loc(e) \cup loc(e') = loc(E) \cup loc(e).$

    \underline{Case 5}: $E = v:=E'$

    $loc(E[e]) = loc(v:=E'[e]) = loc(v) \cup loc(E'[e]) = loc(v) \cup loc(E') \cup loc(e) = loc(E) \cup loc(e)$

    \underline{Case 6}: $E = E';e'$

    $loc(E[e]) = loc(E'[e];e') = loc(E'[e]) \cup loc(e')= loc(E') \cup loc(e) \cup loc(e') = loc(E) \cup loc(e)$

    \underline{Case 7}: $E = v;E'$

    $loc(E[e]) = loc(v;E'[e]) = loc(v) \cup loc(E'[e])= loc(v) \cup loc(E') \cup loc(e) = loc(E) \cup loc(e)$

    \underline{Case 8}: $E = E'~e'$

    $loc(E[e]) = loc(E'[e]~e') = loc(E'[e]) \cup loc(e')= loc(E') \cup loc(e) \cup loc(e')= loc(E) \cup loc(e)$

    \underline{Case 9}: $E = v~E'$

    $loc(E[e]) = loc(v~E'[e]) = loc(v) \cup loc(E'[e]) = loc(v) \cup loc(E') \cup loc(e)= loc(E) \cup loc(e)$

    \underline{Case 10}: $E = \texttt{let x = E' in e'}$

    $loc(E[e]) = loc(\texttt{let x = E'[e] in e'}) = loc(E'[e]) \cup loc(e') = loc(E') \cup loc(e) \cup loc(e')= loc(E) \cup loc(e)$

    \underline{Case 11}: $E = \texttt{let x = v in E'}$

    $loc(E[e]) = loc(\texttt{let x = v in E'[e]}) = loc(v) \cup loc(E'[e]) = loc(v) \cup loc(E') \cup loc(e) = loc(E) \cup loc(e)$

    \underline{Case 12}: $E = (E',e')$

    $loc(E[e]) = loc((E'[e], e')) = loc(E'[e]) \cup loc(e')= loc(E') \cup loc(e) \cup loc(e') = loc(E) \cup loc(e)$

    \underline{Case 13}: $E = (v,E')$

    $loc(E[e]) = loc((v,E'[e])) = loc(v) \cup loc(E'[e]) = loc(v) \cup loc(E') \cup loc(e) = loc(E) \cup loc(e)$

    \underline{Case 14}: $E = \#n~E'$

    $loc(E[e]) = loc(\#n~E'[e]) = loc(E'[e]) = loc(E') \cup loc(e) = loc(E) \cup loc(e).\Box$\\

    \textbf{Claim 2:}  $\langle e, \sigma \rangle \rightarrow \langle e', \sigma' \rangle$ then $dom(\sigma) \subseteq dom(\sigma')$

    \textbf{Proof:} This can be proved by inducting on induction rules. We first prove it for reduction rules followed by context rules.

    \underline{Case 1}: $\langle \texttt{ref v}, \sigma \rangle \rightarrow \langle l, \sigma[v/l] \rangle,~l \notin dom(\sigma)$.

    It's easy to see that $dom(\sigma[v/l]) = dom(\sigma) \cup \{l\} \supseteq dom(\sigma)$.

    \underline{Case 2}: $\langle l~:=~v, \sigma \rangle \rightarrow \langle \texttt{null}, \sigma[v/l] \rangle,~l \in dom(\sigma)$.

    In this case $dom(\sigma[v/l]) = dom(\sigma) \cup \{l\} = dom(\sigma)$.

    \underline{Case 3}:  Other reduction rules:

    $\langle !l, \sigma \rangle \rightarrow \langle \sigma(l), \sigma\rangle$.
    
    $\langle v;e, \sigma \rangle \rightarrow \langle e, \sigma \rangle$.

    $\langle (\lambda x.e)~v, \sigma \rangle \rightarrow \langle e\{v/x\}, \sigma \rangle$.

    $\langle \texttt{let x = v in e}, \sigma \rangle \rightarrow \langle e\{v/x\}, \sigma \rangle$.

    $\langle \#1(v_1, v_2), \sigma \rangle \rightarrow \langle v_1, \sigma \rangle$.

    $\langle \#2(v_1, v_2), \sigma \rangle \rightarrow \langle v_2, \sigma \rangle$.

    $\langle \#n(v_1, v_2), \sigma \rangle \rightarrow \langle \#(n-1)v_2, \sigma \rangle$.

    In the rules above, the store does not change as the expression steps so the claim holds automatically.

    \underline{Case 4}: Context rule:

    \Rule{}{
    \langle e, \sigma \rangle \rightarrow \langle e', \sigma' \rangle
  }{
    \langle E[e], \sigma \rangle \rightarrow \langle E[e'], \sigma' \rangle
  } \\*

    By the induction hypothesis $\langle e, \sigma \rangle \rightarrow \langle e', \sigma' \rangle$ so $dom(\sigma) \subseteq dom(\sigma')$.$\Box$.\\

    \textbf{Claim 3}: If $\langle e, \sigma \rangle \rightarrow \langle e', \sigma' \rangle$ and the following 2 conditions hold for $\langle e, \sigma \rangle$: 

    $loc(e) \subseteq dom(\sigma)$ and $\forall l \in dom(\sigma),~loc(\sigma(l)) \subseteq dom(\sigma)$

    then those 2 conditions also hold for $\langle e', \sigma' \rangle$, that is \\
    $loc(e') \subseteq dom(\sigma')$ and $\forall l \in dom(\sigma')~loc(\sigma'(l)) \subseteq dom(\sigma')$.

    \textbf{Proof}: We will use induction on the derivation of $\langle e, \sigma \rangle \rightarrow \langle e', \sigma' \rangle$. The relation used in this induction proof is the evaluation tree of $\langle e, \sigma \rangle \rightarrow \langle e', \sigma' \rangle$. This is well founded since such a tree is always finite.

    The proof first consider reduction rules and followed by the context rule.

    \underline{Case 1}: $\langle \texttt{ref v}, \sigma \rangle \rightarrow \langle l, \sigma[v/l] \rangle,~l \notin dom(\sigma)$.

    It's easy to see that $l \in dom(\sigma) \cup \{l\} = dom(\sigma[v/l])$.

    Let $l'$ be an arbitrary location in $dom(\sigma[v/l])$. 

    If $l' \neq l$ then $l' \in dom(\sigma)$. By the induction hypothesis $loc(\sigma(l')) \subseteq dom(\sigma) \subseteq dom(\sigma[v/l])$.

    If $l' = l$ then $loc(\sigma[v/l](l)) = loc(v) \subseteq dom(\sigma) \subseteq dom(\sigma[v/l]).$

    \underline{Case 2}: $\langle l~:=~v, \sigma \rangle \rightarrow \langle \texttt{null}, \sigma[v/l] \rangle,~l \in dom(\sigma)$.

    $loc(\texttt{null}) = \varnothing \subseteq dom(\sigma[v/l])$

    Let $l'$ be an arbitrary location in $dom(\sigma[v/l])$. 

    If $l' \neq l$ then by the induction hypothesis $loc(\sigma[v/l](l'))  = loc(\sigma(l')) \subseteq dom(\sigma) = dom(\sigma[v/l])$.

    If $l' = l$ then $loc(\sigma[v/l](l)) = loc(v) \subseteq dom(\sigma) = dom(\sigma[v/l]).$    

    \underline{Case 3}: $\langle !l, \sigma \rangle \rightarrow \langle \sigma(l), \sigma\rangle$.

    Since $l \in dom(\sigma)$ by induction hypothesis $loc(\sigma(l)) \subseteq dom(\sigma)$.

    On the other hand since the store does not change so if $l' \in dom(sigma)$ then $loc(\sigma(l')) \subseteq dom(\sigma).$

    \underline{Case 4}: $\langle v;e, \sigma \rangle \rightarrow \langle e, \sigma \rangle$.

    It's obvious that $loc(e) \subseteq loc(e;v) \subseteq dom(\sigma)$.

    Again the store does not change so if $l' \in dom(sigma)$ then $loc(\sigma(l')) \subseteq dom(\sigma).$

    \underline{Case 5}: Other reduction rules do not change the store. We claim that the locations in the expression stepped into are a subset of locations in the initial expression. This way we can safely conclude that the claim holds here as well.

    $\langle (\lambda x.e)~v, \sigma \rangle \rightarrow \langle e\{v/x\}, \sigma \rangle$.

    $\langle \texttt{let x = v in e}, \sigma \rangle \rightarrow \langle e\{v/x\}, \sigma \rangle$.

    $\langle \#1(v_1, v_2), \sigma \rangle \rightarrow \langle v_1, \sigma \rangle$.

    $\langle \#2(v_1, v_2), \sigma \rangle \rightarrow \langle v_2, \sigma \rangle$.

    $\langle \#n(v_1, v_2), \sigma \rangle \rightarrow \langle \#(n-1)v_2, \sigma \rangle$.

    \underline{Case 6}: The case of context rule can be proved using claim 1 and 2 above.

    \Rule{}{
    \langle e, \sigma \rangle \rightarrow \langle e', \sigma' \rangle
  }{
    \langle E[e], \sigma \rangle \rightarrow \langle E[e'], \sigma' \rangle
  } \\*

    By assumption $loc(E[e]) \subseteq dom(\sigma) \Rightarrow loc(e) \subseteq dom(\sigma)$ since $loc(e) \subseteq loc (E[e])$. Then by induction hypothesis $loc(e') \subseteq dom(\sigma')$.

    Since $loc(E[e']) = loc(E) \cup loc(e')$ and $loc(E) \subseteq dom(\sigma) \subseteq dom(\sigma')$ we conclude that $loc(E[e']) \subseteq dom(\sigma')\Box$.\\

    The fact in the question is an immediate consequence of claim 3. Suppose $\langle e, \{\} \rangle$ steps to $\langle e', \sigma \rangle$ after a number of steps. 

    Let the intermediate expressions and stores be $\langle e, \{\} \rangle \rightarrow \langle e_1, \sigma_1 \rangle \rightarrow \langle e_2, \sigma_2 \rangle \rightarrow \ldots \rightarrow \langle e_n, \sigma_n \rangle \rightarrow \langle e_1, \sigma_1 \rangle$

    Then since $e$ is a valid program $loc(e) = \varnothing \subseteq \{\}$. By applying claim 3 repeatedly we have $loc(e_1) \subseteq dom(\sigma_1) \Rightarrow loc(e_2) \subseteq dom(\sigma_2) \Rightarrow \ldots \Rightarrow loc(e_n) \subseteq dom(\sigma_n) \Rightarrow loc(e') \subseteq dom(\sigma). \Box$
\end{enumerate}
\section{Control Flow}%%%%%%%%%%%%%%%%%%%%%%%%%%%%%%%%%%%%%%%%%%%%%%%%%%%%%%%%%%
\begin{enumerate} [(a)]
    \item We define values in this extension of FL! as follows:

    $v ::= ... | \texttt{abort e}$

    The big step semantics $\Downarrow \in \texttt{Exp} \times \texttt{Env} \rightarrow \texttt{Value} \times \texttt{Env}$ for the extension part of the language can be defined as follows:

  \RuleWithCondition{}{
    \langle e, \sigma \rangle \Downarrow \langle v, \sigma' \rangle
  }{
    \langle \texttt{transaction e}, \sigma \rangle \Downarrow \langle v, \sigma' \rangle
  }{$v \neq \texttt{abort e'}$} \\*

  \Rule{}{
    \langle e, \sigma \rangle \Downarrow \langle \texttt{abort e'}, \sigma' \rangle ~, ~ \langle e', \sigma \rangle \Downarrow \langle v, \sigma'' \rangle
  }{
    \langle \texttt{transaction e}, \sigma \rangle \Downarrow \langle v, \sigma'' \rangle
  } \\*
    \item In order to translate the extended language into FL we will use the store concept in FL!. In FL! each expression is evaluated given a particular store but in this extended language we will use a list of stores which can be recursively defined as a tuple of a store and a stack (list) of stores. The empty stack is equivalent to $(\texttt{null}, \texttt{null})$. The current working store is on the top of the stack.

    Let $\llbracket . \rrbracket_1$ be the translation from FL! to FL. We will define $\llbracket . \rrbracket_1$, the translation from the extended language to FL. Denote $\overline{\sigma}$ as a list of stores so $\# 1 \overline{\sigma}$ returns the head of the list and $\# 2 \overline{\sigma}$ is the tail.
    \begin{align*}
    \llbracket \texttt{transaction e}\rrbracket~\rho~\overline{\sigma} &\triangleq \llbracket e\rrbracket~\rho~(\#1 \overline{\sigma}, \overline{\sigma})\\
    \llbracket \texttt{abort e}\rrbracket~\rho~\overline{\sigma} &\triangleq \llbracket e\rrbracket~\rho~(\#2 \overline{\sigma})\\
    \text{For other expressions }&\text{that are not transaction nor abort:}\\
    \llbracket e\rrbracket~\rho~\overline{\sigma} &\triangleq \texttt{let (x,y) = }\llbracket e\rrbracket_1~\rho~(\#1 \overline{\sigma}) \texttt{ in } (x, (y, \# 2 \overline{\sigma}))
    \end{align*}
    The idea is that when evaluating \texttt{transaction e} we will push the current store onto the top of the list and continue evaluating $e$ in the current store. For \texttt{abort e} we will remove the first store in the list and $e$ will be evaluated in the next store since this is the store of the latest transaction.
    \item We decided to add \texttt{abort e} as one of the values so that when using the big step semantics we are able to evaluate $e$ in the original store.

    Another design choice is made when $\langle \texttt{transaction e}, \sigma \rangle$ is evaluated to $\langle \texttt{abort e'}, \sigma' \langle$ at this point we choose to revert back to the store at the beginning of the transaction ($\sigma$) and evaluate $e'$ in this store to $\langle v, \sigma'' \rangle$. The entire transaction then returns the value of $e'$ together with the store resulted from the evaluation of $e'$ in $\sigma$, that is $\langle v, \sigma'' \rangle$. This choice shows up in the translation to FL when a store is popped from the stack of stores when an expression is evaluated to \texttt{abort e} and at this point the working store which is on the top of the stack is the store at the beginning of the transaction.
    \end{enumerate}
\section{Closure Conversion}%%%%%%%%%%%%%%%%%%%%%%%%%%%%%%%%%%%%%%%%%%%%%%%%%%%%
See \texttt{lifting.zip}.
\section{Debriefing}%%%%%%%%%%%%%%%%%%%%%%%%%%%%%%%%%%%%%%%%%%%%%%%%%%%%%%%%%%%%
\begin{enumerate}
    \item About 40 hours in total.
    \item Somewhere between moderate and difficult.
    \item Yes, both partners participated.
    \item 95\%
    \item
\end{enumerate}
\end{document}
