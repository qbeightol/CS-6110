\section{Hoare Logic}

\newcommand{\true}{\textsf{true}}
\newcommand{\len}[1]{\textsf{len}(#1)}
\newcommand{\dom}{{\sf dom}}
\newcommand{\ifstmt}[3]{{\sf if}~#1~{\sf then}~#2~{\sf else}~#3}
\newcommand{\whilestmt}[2]{{\sf while}~#1~{\sf do}~#2}
\newcommand{\X}{\textsf{X}}

\begin{enumerate}[(a)]
% TODO: consider adding dom_a(X) checks
\item \begin{eqnarray*}
    \dom_a(\bar{n})        & = & {\sf true} \\
    \dom_a(x)              & = & {\sf true} \\
    \dom_a(a_0 \oplus a_1) & = & \dom_a(a_0) \wedge \dom_a(a_1) \\
    \dom_a({\sf X}[a])     & = & 0 \leq a \wedge a < {\sf len(X)} \\
    \dom_a({\sf len(X)})   & = & {\sf true} \\
\end{eqnarray*}

\begin{eqnarray*}
    \dom_b({\sf true})      & = & {\sf true} \\
    \dom_b({\sf false})     & = & {\sf true} \\
    \dom_b(a_0 \odot a_1)   & = & \dom_a(a_0) \wedge \dom_a(a_1) \\
    \dom_b(b_0 \oslash b_1) & = & \dom_b(b_0) \wedge \dom_b(b_1) \\
    \dom_b(\neg b)          & = & \dom_b(b) \\
\end{eqnarray*}

\item
$
\Rule{SimpleAssign}{}{
    \{ \dom_a(a) \wedge P[a / x]\}
    x := a
    \{ P \}
}
$
\\

$
\Rule{ArrayAssign}{}{
    \{ \dom_a({\sf X}[a_0]) \wedge \dom_a(a_1) \wedge P[{\sf X}[a_0 \mapsto a_1] / X \}
    {\sf X}[a_0] := a_1
    \{ P \}
}
$
\\

$
\Rule{If}{
    \{ b \wedge P \}
    c_1
    \{ Q \}
    \\
    \{ \neg b \wedge P \}
    c_2
    \{ Q \}
}{
    \{ \dom_b(b) \wedge P \}
    \ifstmt{b}{c_1}{c_2}
    \{ Q \}
}
$
\\

$\Rule{While}{
    \{ b \wedge P \}
    c
    \{ P \}
}{
    \{ \dom_b(b) \wedge P \}
    \whilestmt{b}{c}
    \{ \neg b \wedge P \}
}$

\item
\begin{enumerate}[i.]
    \item
    \begin{eqnarray*}
        A_0 & = & \forall k .~0 \leq k < j - 1 \Rightarrow  {\sf X}[k] \geq {\sf X}[k + 1] \\
              & & \\
        A_1 & = & \forall k.~0 \leq k < j - 1 \Rightarrow \X[k] \geq \X[k + 1] \\
              & \wedge     & i < \len{\X} \\
              & \wedge     & \forall k .~i < k \leq j \Rightarrow  x \geq X[k]
    \end{eqnarray*}

    \item
    \begin{proof} \hspace{1cm} \\
        \begin{itemize}
        \item $\neg (j < {\sf len(X)})$ implies $j \geq {\sf len}(x)$
        \item we can apply $A_0$ to each $k$ in the range
        $[0, {\sf len}(x) - 2]$ to show that $P$ holds.
        \end{itemize}
    \end{proof}

    \item
    \begin{proof} We'll present the proof tree in three parts: Left-Tree will
        give a derivation for the first command, Right-Tree will give a
        derivation for the second, and an overall derivation will
        show how the proof trees can be combined. We'll present these trees
        in reverse order to reflect the way they were derived.

        \newcommand{\leftPremise}{
            \{ \dom_a(i + 1) \wedge \dom_a({\sf X}[i])
               \wedge A_1[i - 1 / i][i + 1 \mapsto \X[\X[i]
             \}
             ~\X[i = 1] = \X[i]~
             \{ A_1[i - 1/ i] \}
        }

        \newcommand{\leftBase}{
            \{ A_1 \wedge ((i \geq 0) \wedge {\sf X}[i] < x) \}
            ~\X[i = 1] = \X[i]~
            \{ A_1[i - 1/ i] \}
        }

        \newcommand{\rightPremise}{
            \{ \dom_a(i - 1) \wedge A_1[i - 1/ i] \}
            ~i = i - 1~
            \{ A_1 \}
        }

        \newcommand{\rightBase}{
            \{ A_1[i - 1/ i] \}
            ~i = i - 1~
            \{ A_1 \}
        }
        \newcommand{\treeBase}{
            \{ A_1 \wedge ((i \geq 0) \wedge {\sf X}[i] < x) \}
            ~\X[i + 1] = \X[i];~i = i - 1~
            \{ A_1 \}
        }

        \underline{Overall derivation}:

        $$ \inferrule{
                \text{Left-Tree}
                \\
                \text{Right-Tree}
            }{\treeBase}
        $$

        \underline{Right-Tree}: \\
        $$ \inferrule{\rightPremise}{\rightBase} $$

        Note that $\dom_a(i - 1)$ is always true by definition. So any predicate
        $P$ implies $$P~\wedge~\dom_a(i - 1)$$.

        \underline{Left-Tree}:  \\
        $$ \inferrule{\leftPremise}{\leftBase} $$

        This tree is difficult to read, so let's break it into simpler parts.
        The left-most proposition in the premise is the result of applying
        our rule for array assignments. It consists of 3 parts:

        \begin{itemize}
        \item $\dom_a(i + 1)$
        \item $\dom_a({\sf X}[i])$
        \item $A_1[i - 1 / i][\X[i + 1 \mapsto \X[i]]$
        \end{itemize}

        The last clause is equivalent to

        \begin{eqnarray*}
        & & i - 1 < {\sf len(X)} \\
        & \wedge & \forall k. 0 \leq k < j - 1 \Rightarrow \X'[k] \geq \X'[k + 1] \\
        & \wedge & \forall k . i - 1 < k \leq j \Rightarrow  x \geq \X'[k] \\
        \end{eqnarray*}

        where $\X' = \X[i + 1 \mapsto \X[i]]$. These three clauses follow from
        $ A_1 \wedge ((i \geq 0) \wedge {\sf X}[i] < x) $

        \begin{itemize}
        \item $\dom_a(i + 1) = \dom_a(i) \wedge \dom_a(1) = \true $
        \item $\dom_a({\sf X}[i]) = 0 \leq i \wedge i < \len{\X} $ follows from
        $i \geq 0$ and $i < \len{\X}$ (the latter comes from $A_1$).
        \item Note that

        $$i < \len{\X} \Rightarrow i - 1 < \len{\X}$$ \\
        $$ \forall k. (0 \leq k < j - 1) \Rightarrow \X[k] \geq \X[k + 1] $$
        $$ \Rightarrow $$
        $$ \forall k. (0 \leq k < j - 1) \Rightarrow  \X'[k] \geq \X[k + 1] $$ \\
        This is slightly subtle---all but two cases of the conclusion come from the assumption. The
        cases that don't come from the assumption are $k = i$ and $k = i + 1$.
        The former is true because $\X'$ maps $i + 1$ to $\X[i]$ while keeping
        $i's$ binding the same. The latter is true because
        $\X'[i + 1] = \X[i] \leq \X[i + 1] \leq \X[i + 2] = \X'[i + 2]$.

        $$ {\sf X}[i] < x \wedge \forall k .~i - 1 < k \leq j \Rightarrow  x \geq X[k] $$
        $$ \Rightarrow $$
        $$ \forall k .~i < k \leq j \Rightarrow  x \geq  \X'[k] $$

        This implication holds for similar reasons. The key insight is that
        $\X'$ maps $i$ and $i + 1$ to $\X[i]$ which implies $x < \X'[i] = \X'[i + 1]$.
        So this clause holds as well.

        % This clause is just a weakened version of $A_1$. Note that the right
        % clause of the conjunction is the same and that
        % $i < \len{\X} \Rightarrow i - 1 < \len{\X}$, the left clause of the
        % conjunction.

        % $A_1$ satisfies leads directly to most parts of the third clause. What's unclear is whether the $forall k$ proposition holds in the $k = i$ case.

        \end{itemize}




    \end{proof}
    \end{enumerate}


\end{enumerate}
