\documentclass[10pt]{article}
\usepackage{fancyhdr}
\usepackage{geometry}
\usepackage{stmaryrd}
\usepackage{amssymb, amsmath}
\usepackage{enumerate, verbatim}

\pagestyle{fancy}

\lhead{\sf CS 6110 Homework 2}
\rhead{\sf Giang Nguyen (htn26) and Quinn Beightol (qeb2)}


% Commands %%%%%%%%%%%%%%%%%%%%%%%%%%%%%%%%%%%%%%%%%%%%%%%%%%%%%%%%%%%%%%%%%%%%%
\newcommand{\problem}[1]{\subsection*{\sf #1}}
\newcommand{\translate}[2]{\ensuremath {\llbracket #1 \rrbracket}_{#2}}
\newcommand{\defined}{\triangleq}
\renewcommand{\S}{\textsf{S}}
\newcommand{\K}{\textsf{K}}
\newcommand{\I}{\textsf{I}}


\begin{document}
\problem{1 Observational Equivalence}

\textbf{Lemma 1:}
\begin{enumerate}[(i)]
\item \textbf{Claim: } $\equiv_{obs}$ is an equivalence relation assuming that $\equiv$ is also an equivalence relation.

\begin{enumerate}
\item Consider a term $e$. We will prove that $e \equiv_{obs} e$.

For any context $C$, $C[e] \Downarrow$ if and only if $C[e]$ itself converges. In other words,  $C[e] \Downarrow$ iff  $C[e] \Downarrow$.

Suppose $C[e] \Downarrow v_1$ and $C[e] \Downarrow v_2$ under CBV evaluation. Since CBV is deterministic $v_1$ and $v_2$ must be syntactically equivalent. $\equiv$ is reflexive since it is an equivalence relation, $v_1 \equiv v_2$. Therefore $\equiv_{obs}$ is also reflexive.
\item Suppose $e_1 \equiv_{obs} e_2$ we will prove that $e_2 \equiv_{obs} e_1$.

$e_1 \equiv_{obs} e_2$ so $C[e_1] \Downarrow$ iff $C[e_2] \Downarrow \Rightarrow C[e_2] \Downarrow$ iff $C[e_1] \Downarrow$.

Suppose $C[e_1] \Downarrow v_1$ and $C[e_2] \Downarrow v_2$ under CBV evaluation then $v_1 \equiv v_2$ since $e_1 \equiv_{obs} e_2 \Rightarrow v_2 \equiv v_2$ because $\equiv$ is symmetric.

$\Rightarrow e_2 \equiv_{obs} e_1$. $\equiv_{obs}$ is symmetric.
\item Let $e_1, e_2$ and $e_3$ be $3$ terms such that $e_1 \equiv_{obs} e_2$ and $e_2 \equiv_{obs} e_3$.

For any context $C$, $C[e_1] \Downarrow$ if and only if $C[e_2] \Downarrow$ if and only if $C[e_3] \Downarrow$.

Suppose $C[e_1] \Downarrow v_1, C[e_2] \Downarrow v_2$ and $C[e_3] \Downarrow v_3$ then $v_1 \equiv v_2$ and $v_2 \equiv v_3 \Rightarrow v_1 \equiv v_3$ as $\equiv$ is transitive.

Thus $\equiv_{obs}$ is transitive. $\Box$
\end{enumerate}
\item \textbf{Claim}: $\equiv_{obs}$ refines $\equiv$.

Let $v_1$ and $v_2$ be $2$ values such that $v_1 \equiv_{obs} v_2$. Let $C[.]$ be the context $[.]$ that only has a hole.

$C[v_1] = v_1$ and $C[v_2] = v_2$.

$C[v_1] \equiv C[v_2]$ since   $v_1 \equiv_{obs} v_2 \Rightarrow v_1 \equiv v_2 \Box$.
\item This part is exactly the first condition in the definition of $\equiv_{obs}$.
\item Suppose $\equiv^1$ refines $\equiv^2$ and $e_1 \equiv^1_{obs} e_2$. We will prove that $e_1 \equiv^2_{obs} e_2$.

$e_1 \equiv^1_{obs} e_2$ so for any context $C$, $C[e_1] \Downarrow$ if and only if $C[e_2] \Downarrow$.

If $C$ is a context such that $C[e_1] \Downarrow v_1$ and $C[e_2] \Downarrow v_2$ then $v_1 \equiv^1 v_2 \Rightarrow v_1 \equiv^2 v_2 \Box$. 
\end{enumerate}

\textbf{Theorem 2}: For CBV and CBN, $\equiv_{\Downarrow}$ is a fixpoint of the transformation $\equiv$  $\mapsto$  $\equiv_{obs}$  and is the coarsest such fixpoint.

First we will prove that $e_1 \equiv_{\Downarrow} e_2$ if and only if $e_1  (\equiv_{\Downarrow})_{obs} e_2$.

$(\Rightarrow)$ Suppose  $e_1 \equiv_{\Downarrow} e_2$ then by definition for any context $C$, $C[e_1] \Downarrow$ iff $C[e_2] \Downarrow$.

Consider an arbitrary context $C$ such that $C[e_1] \Downarrow v_1$ and $C[e_2] \Downarrow v_2$ under CBV evaluation rules. In fact this entire proof can be used to prove the case for CBN evaluation just by replacing CBV with CBN.

%TODO

$(\Leftarrow)$ This follows directly from (ii) of Lemma 1. $(\equiv_{\Downarrow})_{obs}$ refines $\equiv_{\Downarrow}$.
 %%%%%%%%%%%%%%%%%%%%%%%%%%%%%%%%%%%%%%%%%%
\problem{2 Set Operators and Fixed Points} %%%%%%%%%%%%%%%%%%%%%%%%%%%%%%%%%%%%%
\problem{3 Combinators} %%%%%%%%%%%%%%%%%%%%%%%%%%%%%%%%%%%%%%%%%%%%%%%%%%%%%%%%
\begin{enumerate}[(a)]
  \item
  \item
  \item
  \item
  \item

  % NOTE: this translation doesn't work. Try translating something with a
  % top-level application (ex: Ω). But it might have some salvageable ideas.

  Use $\translate{e}{V}$ to translate a lambda term $e$ in a context
  where the list of variables $V$ are in scope. The order of variables indicate
  when the variable was introduced. E.g. the variables for translating $e$ in
  $\lambda x. \lambda y. e$ would be $V = x, y$)

  \begin{eqnarray*}
    \translate{x}{x, V}         & \defined & \I                  \\
    \translate{x}{y, V}         & \defined & \K~\translate{x}{V} \\
    \translate{\lambda x. e}{V} & \defined & \translate{e}{V,x}  \\
    \translate{e_1~e_2}{V}      & \defined & \S~\translate{e_1}{V}~\translate{e_2}{V}
  \end{eqnarray*}

  To translate closed $\lambda$-terms, use the empty context (denoted $\cdot$).

\end{enumerate}
\problem{4 For Loops} %%%%%%%%%%%%%%%%%%%%%%%%%%%%%%%%%%%%%%%%%%%%%%%%%%%%%%%%%%
\problem{5 Implementing IMP} %%%%%%%%%%%%%%%%%%%%%%%%%%%%%%%%%%%%%%%%%%%%%%%%%%%
\problem{6 Debriefing} %%%%%%%%%%%%%%%%%%%%%%%%%%%%%%%%%%%%%%%%%%%%%%%%%%%%%%%%%
DO NOT FORGET THIS TIMEEEEE

\begin{enumerate}[(a)]
  \item
  \item
  \item
  \item
  \item
\end{enumerate}
\end{document}
